\chapter{Voting Rules}

%%%%%%%%%%%%%%%%%%%%%%%%%%%%%%%%%%%%%%%%%%%%%%%%%%%%%%%%%%%%%%%%%%%%%%%%%%%

\section{Intent}

This chapter was written with the following goals in mind:

\begin{enumerate}
\item Provide clear, unambiguous definitions and procedures for voting
  on general text proposals, errata proposals, and changing this
  document.
\item Enforce a high degree of consensus before text is accepted into
  the MPI standards document.
\item Disallow arbitrary abuse of voting procedures.
\end{enumerate}

This proposal only details {\em official ballot} voting definitions
and procedures.  Unofficial voting procedures, such as ``straw''
votes, are outside the scope of this document.

%%%%%%%%%%%%%%%%%%%%%%%%%%%%%%%%%%%%%%%%%%%%%%%%%%%%%%%%%%%%%%%%%%%%%%%%%%%

\section{Definitions}

\begin{enumerate}
\item {\bf Physical MPI Forum Meeting:} An open meeting of the entire
  MPI Forum in a physical location (vs.\ a teleconference or other
  virtual meeting).  In-person attendance to the meeting is open to
  all organizations in the MPI Forum as well as the general public.

\item {\bf Organization:} A business entity that sends one or more
  representatives to a physical MPI Forum meeting.

\item {\bf Registration:} Individuals register for each physical MPI
  Forum meeting that they will attend.  At the time of registration,
  individuals declare which organization they will represent at that
  meeting.

\item {\bf Overall Organization Eligibility (OOE):} An organization is
  generally eligible to vote if it has registered and had one or more
  representatives physically present at two out of the last three
  physical MPI Forum meetings (including the current meeting).

\item {\bf Individual Meeting Organization Voting Eligibility
    (IMOVE):} An organization is eligible to vote at a specific
  physical MPI Forum meeting if all of the following are true:
  \begin{enumerate}
  \item The organization is OOE.
  \item An individual representing this organization registered for
    that specific physical MPI Forum meeting before the first ballot
    occured.
  \item The organization had at least one of its representatives
    physically present during that specific physical MPI
    Forum meeting.
  \end{enumerate}
  Once an organization becomes IMOVE for a specific physical MPI Forum
  meeting, that organization stays IMOVE for the remainder of that
  specific physical MPI Forum meeting.  For example, if an
  organization's only representative leaves the meeting, that
  organization still remains IMOVE.

\item {\bf Meeting Quorum:} Quorum is established at a physical MPI
  Forum meeting when more than $\nicefrac{2}{3}$ of OOE organizations
  have registered for that meeting.

\item {\bf Individual Ballot Quorum:} Quorum is established for an
  individual ballot when more than $\nicefrac{3}{4}$ of IMOVE
  organizations at the meeting cast a vote (vs.\ abstain).  The number
  of IMOVE organizations is counted at the beginning of each ballot.
\end{enumerate}

%%%%%%%%%%%%%%%%%%%%%%%%%%%%%%%%%%%%%%%%%%%%%%%%%%%%%%%%%%%%%%%%%%%%%%%%%%%

\section{Procedures}

%-------------------------------------------------------------------------

\subsection{Official Ballot Voting}

Official ballot voting and formal readings occur only at physical MPI
Forum meetings where a meeting quorum has been established.

All official ballots must be announced and scheduled at least two
weeks prior to the start date of the physical MPI Forum meeting at
which they will be held.  The dates/times for official ballots will
not change after two weeks prior to the beginning of the meeting to
allow attendees to schedule their travel appropriately.

For each official ballot, each IMOVE organization is individually
polled for their vote.  The designated representative of an IMOVE
organization may vote ``yes,'' vote ``no,'' or abstain from voting.
Proxies are not permitted.  If no representative of an IMOVE
organization is physically present at the time of the ballot, that
organization has implicitly abstained.

A ballot passes if:

\begin{enumerate}
\item The ballot meets the requirements for the individual ballot
  quorum, and
\item The number of ``yes'' votes is more than $\nicefrac{3}{4}$ of
  the sum of ``yes'' and ``no'' votes.
\end{enumerate}

\begin{rationale}
  The first condition prevents large numbers of abstentions from
  skewing results.  The second condition sets a high requirement for
  consensus before a ballot will pass.
\end{rationale}

Note that if a ballot fails to meet the required individual ballot
quorum, the ballot can be re-cast one time at the same physical MPI
Forum meeting.  The ballot may also be deferred to a subsequent
physical MPI Forum meeting.  Specifically: failing to establish the
individual ballot quorum does not mean that the ballot failed.

%-------------------------------------------------------------------------

\subsection{General Text Proposals}

General text proposals for the MPI standards documents (including
so-called ``companion documents'', such as the MPIR specification
document) are usually ``not trivial'' changes, and typically add new
semantics, change or clarify existing semantics, or remove
previously-defined semantics.

General text proposals use the following process to be accepted into
an MPI standards document:

\begin{enumerate}
\item Have a formal reading at a physical MPI Forum meeting where the
  meeting quorum has been met.
  \begin{enumerate}
  \item The final text of the proposal to be read must be made
    publicly available via the general MPI Forum broadcast email list
    at least two weeks prior to the start date of the physical MPI
    Forum meeting at which it is to be formally read.
  \item The formal reading must be scheduled on the physical MPI Forum
    meeting's agenda at least two weeks prior to the meeting's start
    date.
  \item There is no criteria for ``passing'' or ``failing'' a formal
    reading.  It is up to the proposal's author(s) to decide whether
    to bring the proposal up for a formal ballot at a subsequent
    meeting.
  \end{enumerate}

\item Pass a first official ballot at a physical MPI Forum meeting.
  \begin{enumerate}
  \item A proposal's first ballot can only be conducted after its
    formal reading.
  \item A proposal's first ballot must be conducted at a different
    physical MPI Forum meeting than which it was formally read.
  \end{enumerate}

\item Pass a second official ballot at a physical MPI Forum meeting.
  \begin{enumerate}
  \item A proposal's second ballot can only be conducted after its
    first ballot passes.
  \item A proposal's second ballot must be conducted at a different
    physical MPI Forum meeting than which it passed its first ballot.
  \end{enumerate}

\item Changes to proposal text after it was made available for the
  formal reading (i.e., at least two weeks prior to the start date of
  the physical MPI Forum meeting at which it was read) are permitted
  in some cases:
  \begin{enumerate}
  \item Before the second ballot, changes are permitted if the text
    delta is presented at a physical MPI Forum meeting and approved
    via a special formal ballot of IMOVE organizations at that
    meeting:
    \begin{enumerate}
    \item The ballot meets the requirements for the individual
      ballot quorum, and
    \item There are zero ``no'' votes.
    \end{enumerate}
    
    \begin{rationale}
      The first condition prevents a large number of abstentions.
      The second condition ensure that all non-abstaining
      organizations are unanimous in their consent of the text
      changes.
    \end{rationale}
    
    If the special ballot fails, the original text of the proposal
    is used.

  \item After the second ballot, text changes that do not change the
    semantics of the proposal are permitted with the unanimous consent
    of the relevant chapter committee(s).
  \end{enumerate}
\end{enumerate}

Proposals may be voluntarily withdrawn at any time before the second
ballot passes.

Ballots may be deferred to a subsequent physical MPI Forum meeting in
the following cases:

\begin{enumerate}
\item Before the ballot is conducted, the proposal author requests a
  deferral to the next physical MPI Forum meeting.
\item When the ballot is conducted, it fails to meet the individual
  ballot quorum.
\end{enumerate}

If a proposal fails either of its ballots, or if a proposal is
withdrawn, it must perform the entire procedure again (i.e., start
over with a formal reading).  If either ballot fails to establish its
per-ballot quorum, it may be re-cast within the timeframes specified
above.

%-------------------------------------------------------------------------

\subsection{Errata Proposals}

Errata proposals for the MPI standards documents are usually ``small''
and deal with ``critical'' changes to documents to correct errors,
clarify egregious ambiguities, etc.

Errata proposals use the following process to be accepted into an MPI
standards document:

\begin{enumerate}

\item Pass a single official ballot at a physical MPI Forum meeting.
  \begin{enumerate}
  \item Final errata proposal text must be made publicly available by
    the Errata document editor via the general MPI Forum broadcast
    email list at least two weeks prior to the start date of the
    physical MPI Forum meeting at which its ballot will occur.
  \end{enumerate}

\item Changes to proposal text after it was made available (i.e., at
  least two weeks prior to the start date of the physical MPI Forum
  meeting at which it was balloted) are permitted in some cases:
  \begin{enumerate}
  \item Before the ballot, changes are permitted if the text delta is
    presented at a physical MPI Forum meeting and approved via a
    special formal ballot of IMOVE organizations at that meeting:
    \begin{enumerate}
    \item The ballot meets the requirements for the individual
      ballot quorum, and
    \item There are zero ``no'' votes.
    \end{enumerate}
    
    If the special ballot fails, the original text of the errata
    proposal is used.
  \item After the ballot, text changes that do not change the
    semantics of the proposal are permitted with the unanimous consent
    of the relevant chapter committee(s).
  \end{enumerate}
\end{enumerate}

If an errata proposal fails its ballot, it must perform the entire
procedure again (i.e., start over by posting the text a minimum of two
weeks before a physical MPI Forum meeting).

Procedures for deferring errata proposal ballots are the same as
those for general text proposals.


%-------------------------------------------------------------------------

\subsection{Changing These Rules}

The procedure for changing these rules is essentially the same as for
Errata Proposals: publish the proposed change at least two weeks prior
to a physical MPI Forum meeting and then pass one official ballot.

% LocalWords:  OOE IMOVE
